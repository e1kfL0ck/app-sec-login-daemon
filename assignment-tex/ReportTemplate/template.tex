\documentclass[11pt]{report}
\usepackage[utf8]{inputenc}
\usepackage[T1]{fontenc}
\usepackage{lmodern}
\usepackage[margin=2.5cm]{geometry}
\usepackage{hyperref}
\usepackage{enumitem}
\usepackage{parskip}
\usepackage{svg}
\usepackage{float}
\usepackage{titlesec}

% -------------------------------------------------------------------
% CUSTOM FORMATTING
% -------------------------------------------------------------------
% This removes "Chapter X" and the number, but keeps the Title
\titleformat{\chapter}[hang] 
{\normalfont\huge\bfseries}{}{0pt}{}

% -------------------------------------------------------------------
% TITLE PAGE INFORMATION
% -------------------------------------------------------------------
\title{Course Name -- Laboratory Report\\[0.4em]
\large LAB-XX: Title of the Specific Lab Exercise}

% Update Authors here
\author{Student One Name \\ \texttt{student1.email@student.put.poznan.pl} \\
Student Two Name \\ \texttt{student2.email@student.put.poznan.pl} \\
\\
Supervisor: Supervisor Name, Ph.D. \\ Politechnika Poznańska}

\date{\today}

% -------------------------------------------------------------------
% DOCUMENT START
% -------------------------------------------------------------------
\begin{document}

% Generates the Title Page
\maketitle

% Generates the Table of Contents
\tableofcontents
\newpage

% -------------------------------------------------------------------
% INTRODUCTION
% -------------------------------------------------------------------
\chapter*{Introduction}
\addcontentsline{toc}{chapter}{Introduction} % Adds Intro to TOC manually since it's unnumbered

Type your introduction here. Briefly describe the goal of the laboratory, the technologies used, and the scope of the report.

% -------------------------------------------------------------------
% PART 1
% -------------------------------------------------------------------
\part{General Concepts}

\chapter{Guiding Principles}
This section describes the theoretical approach or general rules followed during the lab.

\section{First Principle (e.g., Input Validation)}
Describe the first principle here.
\begin{enumerate}
    \item \textbf{Point One}: Description of the first step.
    \item \textbf{Point Two}: Description of the second step.
    \item \textbf{Point Three}: Description of the third step.
\end{enumerate}

\section{Second Principle}
Describe the methodology or lifecycle used.
\begin{itemize}
    \item \textbf{Phase 1}: Detail.
    \item \textbf{Phase 2}: Detail.
    \item \textbf{Phase 3}: Detail.
\end{itemize}

% -------------------------------------------------------------------
% PART 2
% -------------------------------------------------------------------
\part{Technical Implementation}

\chapter{Module Name (e.g., Core Routing)}
Description of this specific module.

\section{Route Surface}
List the endpoints or routes involved.
\begin{itemize}[leftmargin=1.5em]
    \item \texttt{/example/route} -- Public; description of behavior.
    \item \texttt{/admin/route} -- Authenticated; protection mechanism.
\end{itemize}

\section{Security Controls}
\begin{itemize}[leftmargin=1.5em]
    \item \textbf{Control Name}: Description of the implementation.
    \item \textbf{Another Control}: Description. 
    
    \textit{Note: You can use italic text for important notes or observations regarding the implementation.}
\end{itemize}

\chapter{Another Module (e.g., Authentication)}

\section{Route Surface}
\begin{itemize}[leftmargin=1.5em]
    \item \texttt{/login} -- Entry point.
    \item \texttt{/register} -- Entry point.
\end{itemize}

\section{Implementation Details}
\begin{itemize}[leftmargin=1.5em]
    \item \textbf{Feature A}: Explain how it works.
    \item \textbf{Feature B}: Explain how it works.
    \item \textbf{Feature C}: Explain how it works.
\end{itemize}

% -------------------------------------------------------------------
% CONCLUSION (Optional)
% -------------------------------------------------------------------
\chapter*{Conclusion}
\addcontentsline{toc}{chapter}{Conclusion}

Summary of findings and final thoughts.

\end{document}