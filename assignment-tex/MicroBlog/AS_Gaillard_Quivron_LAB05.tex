\documentclass[11pt]{report}
\usepackage[utf8]{inputenc}
\usepackage[T1]{fontenc}
\usepackage{lmodern}
\usepackage[margin=2.5cm]{geometry}
\usepackage{hyperref}
\usepackage{enumitem}
\usepackage{parskip}
\usepackage{svg}
\usepackage{float}

\title{Application Security -- Laboratories\\[0.4em]
\large LAB-05}
\author{Gaillard Théo \\ \texttt{theo.gaillard@student.put.poznan.pl} \\
Quivron Emile \\ \texttt{emile.quivron@student.put.poznan.pl} \\
\\
Supervisor: Michał Apolinarski, Ph.D. \\ Politechnika Poznańska}
\date{\today}

\begin{document}
\maketitle
\tableofcontents
\newpage

This report documents the implementation of the User Registration Module as part of a MicroBlog application. The MicroBlog application will be discussed extensively here.
\part{API modifications and encapsulation}

\chapter{Component Overview}

\part{MicroBlog Application Implementation}

\chapter{New Features Introduced in Commit 559a80d}
This section documents the features added on the main branch by merge commit \texttt{559a80d}. The comparison is against
the main branch state immediately before that merge, so the list focuses on new functionality and structural changes
introduced by the merge.

\section{Application Structure and Modularity}
\subsection*{Blueprint-based routing}
Blueprints split the application into three coherent domains: authentication (\texttt{auth}), content management
(\texttt{content}), and user management (\texttt{user}). Each blueprint owns its URL prefix, templates, and route
handlers, which keeps domain-specific logic localized and makes it easier to test, document, and evolve each area
independently. This structure also simplifies URL mapping by making ownership explicit (e.g., \texttt{/register} under
auth, \texttt{/content/*} for posts and feeds, \texttt{/user/*} for profiles and settings).

\subsection*{Service and repository layers}
The new service layer (\texttt{services.py}) centralizes business rules such as login validation, content visibility, and
account state handling. The repository layer (\texttt{repository.py}) encapsulates all database queries and mutations.
This separation reduces direct SQL in route handlers, prevents duplicated logic across endpoints, and allows future
changes (database schema or backend storage) to be made in a single place with minimal impact on routing code.

\subsection*{Validation utilities}
Input validation is now consolidated in dedicated \texttt{validators.py} modules for each domain. These validators
enforce format, length, and policy rules (email format, password strength, post length, etc.) before data reaches the
service and repository layers. Centralizing validation improves consistency across features, reduces the risk of missing
checks in new endpoints, and documents input expectations in a single, reusable location.

\subsection*{Session helpers}
Shared decorators and utilities in the session helper module provide consistent enforcement of authentication and admin
authorization. They abstract common access checks (logged-in requirement, admin-only requirement, already-logged-in
redirects) so that routes remain concise and security controls are applied uniformly across the application.

\section{Authentication and Account Security}
\subsection*{Account activation flow}
New registrations create an activation token stored in the database with expiry metadata. The activation endpoint
validates the token, checks expiration and usage status, and activates the account only when the token is valid. This
introduces a clear account lifecycle (registered $\rightarrow$ activated) and enables email-based verification that
prevents inactive accounts from accessing protected features.

\subsection*{Password reset workflow}
Password resets use one-time tokens with expiry checks, generated on request and stored in the token table. The reset
process validates the token before allowing a password change, then marks the token as used. The UI always returns a
generic success message to avoid email enumeration, while still allowing legitimate users to complete the reset flow
securely.

\subsection*{Multi-factor authentication}
MFA support adds a setup flow that generates a TOTP secret, renders a QR code for authenticator apps, and confirms
time-based one-time codes. On successful setup, the system stores the secret and issues a set of backup codes for
recovery. The verification flow accepts either a valid TOTP code or an unused backup code, then finalizes login and
rotates backup codes as they are consumed.

\subsection*{Login protections}
Authentication now tracks failed login attempts and enforces a reset requirement after repeated failures. On successful
login, the session is cleared to mitigate session fixation, and last-login timestamps are updated when MFA is not in
play. These measures improve resilience against brute-force attacks and reduce the likelihood of session reuse.

\subsection*{Admin disablement}
Administrative controls allow privileged users to disable accounts at the database level. Disabled-by-admin accounts are
prevented from logging in and cannot be re-enabled by the affected user, ensuring that security or policy violations can
be enforced centrally.

\section{MicroBlog Content Features}
\subsection*{Public feed and admin feed}
The content module introduces a public feed that lists all public posts with pagination. Administrators are granted a
separate global feed that surfaces all posts, including private content, which supports moderation and oversight duties.
These two feeds expose different views of the same underlying dataset based on role.

\subsection*{Post management}
Users can create, edit, and delete posts, selecting whether each post is public or private. Editing flows re-load
current post data and attachments, while deletion enforces ownership or admin privileges. Viewing a post respects
visibility rules and displays related comments and attachments.

\subsection*{Comments and attachments}
Posts can be commented on by authenticated users, and attachments can be uploaded alongside content creation or edits.
Attachment downloads are served through permission checks and safe file delivery, preserving original filenames while
ensuring storage safety and access control.

\subsection*{Search}
Keyword search scans content and returns matching posts with clear empty-state feedback. The route validates input,
returns errors when necessary, and ensures search results are displayed consistently alongside the entered query.

\subsection*{Content permissions}
Authorization checks protect content operations. View permissions respect public/private status and consider disabled
authors; edit and delete actions are limited to post owners or admins. This isolates security rules from route handlers
and provides a single location for permission logic.

\section{User Profiles and Administration}
\subsection*{Profile pages}
Each user now has a profile page that displays account information and a personalized feed of their posts. For admins,
the same template can be used to view other users' profiles, supporting oversight and account review tasks without
introducing a separate UI path.

\subsection*{Account settings}
The settings page consolidates self-service actions, including email updates, account deletion, and self-disable or
reactivation flows. Each action validates the current password and requires explicit confirmation to reduce accidental
changes, while success and error messages are displayed on the same page for clear feedback.

\subsection*{Admin user management}
Administrative tooling provides a paginated user list and dedicated controls to disable or re-enable accounts. Admins
can also view a user's profile and posts to audit behavior. Safeguards prevent administrators from disabling their own
accounts.

\section{UI and Template Coverage}
\subsection*{New templates}
The merge introduces a full template set for the content and user modules, including feeds, post creation and editing,
search results, profile pages, settings, and admin user views. These templates establish consistent layouts for new
features and reduce ad hoc rendering logic in routes.

\subsection*{Updated base and dashboard}
The base layout is adjusted to accommodate new navigation targets and maintain consistent styling across the expanded
feature set. The dashboard template is updated to reflect the broader application scope and quick actions now available
to authenticated users.

\subsection*{Static assets}
A favicon is added for browser identification and branding, while content-specific CSS augments the existing styling to
support the new templates and layout structures.

\section{Database and Deployment Updates}
\subsection*{Extended schema}
The database schema expands beyond authentication to include posts, comments, and attachments, enabling full MicroBlog
functionality. The users table gains role data, disabled-state flags, MFA secrets, and backup codes, which are required
for admin control and two-factor authentication flows.

\subsection*{Initialization routines}
Database initialization now includes optional seed data for development and demonstrations. When debug mode is enabled,
the system creates a test user, an admin account, and example posts, allowing quick verification of the content and
authentication features without manual setup.

\subsection*{Environment and runtime}
Container and runtime configuration is updated to align with the new module layout and extended application scope.
Environment examples, the entrypoint script, and Docker-related files reflect the additional dependencies and routes
introduced by the merge.

\end{document}
